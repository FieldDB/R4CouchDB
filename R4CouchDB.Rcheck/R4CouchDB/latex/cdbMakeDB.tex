\HeaderA{cdbMakeDB}{Creates a new database}{cdbMakeDB}
\keyword{misc}{cdbMakeDB}
\begin{Description}\relax
The name of the new database is taken from \code{cdb\$newDBName}.
\end{Description}
\begin{Usage}
\begin{verbatim}
cdbMakeDB(cdb)
\end{verbatim}
\end{Usage}
\begin{Arguments}
\begin{ldescription}
\item[\code{cdb}] The \code{cdb} have to provide \code{cdb\$serverName},
\code{cdb\$port} and  \code{cdb\$newDBName}

\end{ldescription}
\end{Arguments}
\begin{Details}\relax
The work is done by \code{getURLContent()}. After creating the new database
the function makes the shortcut \code{cdb\$DBName <- cdb\$newDBName} so that further
operations happen on the new created database.
\end{Details}
\begin{Value}
\begin{ldescription}
\item[\code{cdb}] The CouchDB answer is stored in \code{cdb\$res}. Any
problems on the R side are reportet in \code{cdb\$error}

\end{ldescription}
\end{Value}
\begin{Note}\relax
The convention for database naming should be implemented.
\end{Note}
\begin{Author}\relax
wact.b.prot
thsteinbock@web.de
\end{Author}
\begin{References}\relax
\url{  http://couchdb.apache.org/       }
\end{References}
\begin{SeeAlso}\relax
\code{cdbRemoveDB}
\end{SeeAlso}

