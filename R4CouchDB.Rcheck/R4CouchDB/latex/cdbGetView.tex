\HeaderA{cdbGetView}{Receive view results from CouchDB}{cdbGetView}
\keyword{misc}{cdbGetView}
\begin{Description}\relax
The function provides accesses to CouchDB views.
\end{Description}
\begin{Usage}
\begin{verbatim}
cdbGetView(cdb)
\end{verbatim}
\end{Usage}
\begin{Arguments}
\begin{ldescription}
\item[\code{cdb}] Beside the connection details
(\code{cdb\$port},\code{cdb\$DAName} ...) the \code{cdb\$design} and
\code{cdb\$view} is needed.


\end{ldescription}
\end{Arguments}
\begin{Details}\relax
Query params e.g. \code{"reduce=false"} or  \code{"group\_level=1"} can
be provided in \code{cdb\$queryParam}
\end{Details}
\begin{Value}
\begin{ldescription}
\item[\code{cdb }] The result of the request is stored in cdb\$res after
converting the json answer into a list using fromJSON(). If a needed
cdb list entry was not provided cdb\$error says something about the R
side


\end{ldescription}
\end{Value}
\begin{Note}\relax
For the moment only one \code{cdb\$queryParam} is possible. In the future
maybe Duncans \code{RJavaScript} package can be used to generate views
without leaving R.
\end{Note}
\begin{Author}\relax
wact.b.prot
thsteinbock@web.de
\end{Author}
\begin{References}\relax
\url{  http://www.omegahat.org/RCurl/        }
\url{  http://www.omegahat.org/RJSONIO/      }
\url{  http://couchdb.apache.org/            }
\url{  http://www.omegahat.org/RJavaScript/  }
\end{References}
\begin{Examples}
\begin{ExampleCode}
## Not run:
## If:
## function(doc) {
##  emit(null, doc._id);
## }
## is stored under design: test, view: id
## in the database db5r

cdb <- cdbIni()
cdb$DBName <- "db5r"
cdb$design <- "test"
cdb$view <- "id"

## cdbGetView(cdb)$res

\end{ExampleCode}
\end{Examples}

