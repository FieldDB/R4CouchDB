\HeaderA{cdbUpdateDoc}{This function  updates an existing doc. This
essentially means that a revision, corresponding to the '\_id'
has to be provided (in what way ever).}{cdbUpdateDoc}
\keyword{misc}{cdbUpdateDoc}
\begin{Description}\relax
Updating a doc at couchdb means executing a http "PUT" request. The  \code{cdb}
list must contain the \code{cdb\$serverName}, \code{cdb\$port},
\code{cdb\$DBName}, \code{cdb\$id}. If
the \code{cdb\$rev} is not given the function uses \code{cdbGetDoc()} to receive the
latest revision number.
\end{Description}
\begin{Usage}
\begin{verbatim}
cdbUpdateDoc(cdb)
\end{verbatim}
\end{Usage}
\begin{Arguments}
\begin{ldescription}
\item[\code{cdb}] the cdb connection configuration list must contain the \code{cdb\$serverName},
\code{cdb\$port},  \code{cdb\$DBName} and  \code{cdb\$id}. The data which
updates the data stored in the doc is provided in \code{cdb\$dataList}

\end{ldescription}
\end{Arguments}
\begin{Details}\relax
\code{getURLContent()} with \code{customrequest = "PUT"} does the work.
\end{Details}
\begin{Value}
\begin{ldescription}
\item[\code{cdb }] The result of the request is stored in \code{cdb\$res} after
converting the answer by means of  \code{fromJSON()}. If a needed
\code{cdb\$} list entry is not provided \code{cdb\$error} maybe says
something about the R  side.

\end{ldescription}
\end{Value}
\begin{Author}\relax
wact.b.prot
thsteinbock@web.de
\end{Author}
\begin{References}\relax
\url{  http://www.omegahat.org/RCurl/   }
\url{  http://www.omegahat.org/RJSONIO/ }
\url{  http://couchdb.apache.org/       }
\end{References}
\begin{SeeAlso}\relax
\code{cdbInit()}
\end{SeeAlso}

